\section{Introduction}
\label{introduction}
%\subsection{Why does this document exist?}

Knowledge representation is becoming increasingly important in view of the large amounts of data that are handled in the modern scientific landscape. Many of the domains that have most readily realised this problem and worked on potential solutions have been the domain of biochemistry. And although the developments sought here were not accompanied by philosophers, this process organically led to the development of formal and structured representations of certain domains. While the resulting structures were not the first formal ontologies, they are ones that are still in wide use to this day.
\todo{FN: this is a misconception. ontologies predate their use in biology. the earliest ontology that I am aware of was developed in the late 60s. By the 80s ontologies were used in major projects on AGI (CYC) or for natural language generation. MG: Added some hedging.}
These approaches have led to major advances in the organisation, structuring and communication of scientific results. Since then, a variety of other domains have tried to adapt a similar process and develop their own ontologies. However, the development of ontologies from the domain of biochemistry was the result of a years-long process that also involved a large number of errors and course corrections. One of the greatest challenges is also one of the greatest strengths of ontologies: Interoperability with other ontologies. To ensure this interoperability, ontologies must follow certain principles. In the field of biochemistry, the OBO Foundry has established itself, which offers functionalities for a rich network of ontologies from the domain, but at the same time also defines rules.

The purpose of this document is to define a similar set of rules for open ontology development, but which addresses a broader domain and at the same time lowers the barrier of entry for new ontology developers. To this end, we will outline a workflow that can be used to build new ontologies more efficiently. This workflow is based not only on our own years of experience in ontology development, but also on the rules of external experts such as the OBO Foundry.

\subsection{Introductory literature}


These guidelines are based on a variety of different sources and experiences. In this section we would like to present some of the important works on which this document is based.

%The following workflow is largely inspired by an infrastructure described by Allemang et. al at FOIS 2021 \cite{kendall-workflow} and the methods used to develop large ontologies such as (FIBO) \cite{fibo2013} and the Open Energy Ontology (OEO) \cite{oeo2021}.

\begin{itemize}
    \item The OWL 2 Primer gives a comprehensive overview of all features of the widely used OWL 2 language. \cite{owlprimer}
    \item Maria Keet is a recognized expert in the field of ontology development. Her recent book provides a good overview of the topic. \cite{keet2018introduction}
    \item There are a variety of views on what ontologies are and how the process of ontology development should be understood. This guide is based on the principles described by Neuhaus et al. \cite{Neuhaus2022OntologyDI, neuhaus2018ontology}
    \item Our ontologies are based on the Basic Formal Ontology (BFO). This useful structuring of fundamental ontological concepts is described in Barry Smith's corresponding book \cite{bfo-book}
\end{itemize}

\subsection{Related efforts}

This document is the result of a collaboration between various projects from a wide range of institutions. We would like to take this opportunity to briefly present these individual projects.

\paragraph{Open Energy Family}

With the aim of making energy research more FAIR\footnote{\url{https://www.go-fair.org/fair-principles/}}, the Open Energy Family provides tools and infrastructure for knowledge exchange in the domain of energy system modeling. In the course of several research projects\footnote{\url{https://www.iee.fraunhofer.de/de/projekte/suche/2021/sirop.html}}\footnote{\url{https://reiner-lemoine-institut.de/szenariendb/}}\footnote{\url{https://reiner-lemoine-institut.de/open_ego-open-electricity-grid-optimization/}} funded by the German Federal Ministry For Economic Affairs And Climate Action, a variety of features have been developed with which experts from the domain can publish, communicate and compare their results more efficiently, transparently and interoperably. For example, research data can be published on the Open Energy Platform\footnote{\url{https://openenergyplatform.org/about/}}, annotated there using the Open Energy Ontology\footnote{\url{https://openenergyplatform.org/ontology/}} (OEO, \cite{oeo2021}). Via the Open Energy Knowledge Graph, datasets can be linked to energy scenarios and model factshhets and become queriable.

\paragraph{Platform MaterialDigital (PMD)}

The Platform MaterialDigital (PMD)\footnote{\url{https://materialdigital.de/}} is a collaborative project funded by the Federal Ministry of Education and Research that aims at bringing together and supporting interested parties from the industrial and academic sectors in the implementation of digitization tasks for materials in the long term.

For this purpose, a virtual material data space is supposed to be created and the handling of hierarchical, process-dependent material data is to be systematized. The development of and finding agreements on data structures and interfaces that are implemented in specific software tools and offer users of PMD easy access and specific added value in their own projects are key aspects of the work. As part of this work addressing semantic interoperability of data, ontologies are created within PMD. Furthermore, the procedures of ontology development, extension, maintenance and curation are to be made accessible to a wide range of interested parties, e.g., domain experts from the field of materials science and engineering (MSE), with a low entry threshold. 
As part of this strategy, the PMD Core Ontology (PMDco)\footnote{\url{https://github.com/materialdigital/core-ontology}} was created by PMD as a mid-level ontology aimed at promoting MSE domain interoperability. Developed through continuous collaboration with the MSE community, the PMDco provides a selection of essential domain key terms within an intermediate semantic layer that is easily understandable and usable to MSE experts. Moreover, several application ontologies were created that are supposed to be used as best practice example of domain ontology development. The following publications provide more information on the PMD work: 

\begin{itemize}
    \item Bayerlein et al., \href{https://doi.org/10.1002/adem.202101176}{A Perspective on Digital Knowledge Representation in Materials Science and Engineering}: \cite{BayerleinPerspective2022}.
    \item Bayerlein et al., \href{https://doi.org/10.1016/j.matdes.2023.112603}{PMD Core Ontology: Achieving semantic interoperability in materials science}: \cite{BayerleinPMDco2024}
    \item Schilling et al., \href{https://doi.org/10.1002/adem.202400138}{FAIR and structured data: A domain ontology aligned with standard-compliant tensile testing}: \cite{SchillingTTO2023}
\end{itemize}

\newpage
