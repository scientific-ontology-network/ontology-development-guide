
\subsection{Development Rules}
\label{ssec:development-rules}

\todo{(MG) this should be integrated in the previous chapter.}
For the consistent development of ontologies, a few basic rules and principles have to be followed.
Therefore, a selected and adapted list of principles based on the development principles\footnote{\url{https://obofoundry.org/principles/fp-000-summary.html}} of OBO Foundry\cite{smith2007obo} is used:

\begin{itemize}
    \item \textbf{URI/Identifier Space} - Each ontology MUST have a unique IRI in the form of a permanent URL. \\ \textit{For the PMDco ontologies family, the namespace https://w3id.org/pmd/co is used}
    \item \textbf{Scope} - The scope of the ontology and content that adheres to that scope is clearly specified. \\ \textit{For the PMDco ontologies family, the scopes are defined HERE [INSERT URL?]}
    \item \textbf{Textual Definitions} - The ontology has textual definitions for its classes.
    \item \textbf{Documentation} - The owners of the ontology should strive to provide as much documentation as possible. \\ \textit{For the PMDco ontologies family, comprehensive documentation can be found in the corresponding {\github}  repository: \url{https://github.com/materialdigital/core-ontology}}
    \item \textbf{Collaboration} - Ontology development should be carried out in a collaborative fashion.
    \item \textbf{Naming Conventions} - The names (primary labels) for elements (classes, properties, etc.) in an ontology must be intelligible to scientists and amenable to natural language processing.
    \item \textbf{Responsiveness} - Ontology developers shall offer channels for community participation and be responsive to requests.
\end{itemize}





