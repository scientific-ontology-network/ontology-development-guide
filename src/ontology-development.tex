
\section{The PMD ontology design process}

\subsection{Ontology development guidelines}

\subsection{Development Rules}
For the consistent development of ontologies, a few basic rules and principles have to be followed.
Therefore, a selected and adapted list of principles based on the development principles\footnote{\url{https://obofoundry.org/principles/fp-000-summary.html}} of OBO Foundry\cite{smith2007obo} is used:

\begin{itemize}
    \item \textbf{Open} - The ontology MUST be openly available to be used by all without any constraint.
    \item \textbf{Common Format} - The ontology is made available in a common formal language in an accepted concrete syntax.
    \item \textbf{URI/Identifier Space} - Each ontology MUST have a unique IRI in the form of a permanent URL. \\ \textit{For the PMDco ontologies family, the namespace https://w3id.org/pmd/co is used}
    \item \textbf{Versioning} - The ontology provider has documented procedures for versioning the ontology, and different versions of ontology are marked, stored, and officially released.
    \item \textbf{Scope} - The scope of the ontology and content that adheres to that scope is clearly specified. \\ \textit{For the PMDco ontologies family, the scopes are defined HERE [INSERT URL?]}
    \item \textbf{Textual Definitions} - The ontology has textual definitions for its classes.
    \item \textbf{Documentation} - The owners of the ontology should strive to provide as much documentation as possible. \\ \textit{For the PMDco ontologies family, comprehensive documentation can be found in the corresponding {\github}  repository: \url{https://github.com/materialdigital/core-ontology}}
    \item \textbf{Collaboration} - Ontology development should be carried out in a collaborative fashion.
    \item \textbf{Authority} - There should be an authority responsible for communications between the community and the ontology developers, for mediating discussions involving maintenance in the light of scientific advance, and for ensuring that all user feedback is addressed.
    \item \textbf{Naming Conventions} - The names (primary labels) for elements (classes, properties, etc.) in an ontology must be intelligible to scientists and amenable to natural language processing.
    \item \textbf{Responsiveness} - Ontology developers shall offer channels for community participation and be responsive to requests.
\end{itemize}

\subsection{Ontology Design Patterns}

The PMD design process is informed by the PMD Semantic Patterns. These patterns describe general design patterns that can be used to model similar concepts in a similar fashion. 

Example(s): 
\begin{itemize}
    \item OTTR\footnote{\url{https://ottr.xyz/}}
\end{itemize}

\subsection{Usage of external ontologies}

Example(s): 
\begin{itemize}
    \item ChEBI\footnote{\url{https://www.ebi.ac.uk/chebi/}}
    \item QUDT\footnote{\url{https://qudt.org/}}
\end{itemize}

\subsection{Using {\github}  for ontology development}

Ontology development unfolds across two primary platforms: {\github} issues and online meetings. {\github}  serves as the central hub, where issues act as the primary conduit for external feedback, extension requests, and developmental discussions. Online meetings are strategically employed for tackling challenging and contentious definitions. The overarching objective in both {\github}  and meetings is to reach a consensus among all participating experts, ensuring a robust and collaborative ontology development process.

In the following, we propose a general workflow from a user request to the final extension of the ontology.

\subsubsection{Reporting missing classes/properties/axioms}
\todo{MS: Add a note on the extend of the issue topic: ideally create atomic issues (1 term = 1 issue), however, this is sometimes not feasible, because some terms and relations have to be discussed in context with others. for a single issue max a couple (1-5) of related terms/relations. In OEO we also use meta-issues for broader topics, that can be split / refer to atomic issues.}
\begin{enumerate}
    \item Open a new issue at \url{https://github.com/materialdigital/\fakeontologyname}
    \item Name the issue in such a way that the issue title makes sense if read after "The issue is that ..."
    \begin{itemize}
        \item Good: "Microscopes are not covered", "The definition of iron is inconsistent"
        \item Bad: "Can we cover microscopes?", "faulty definition of iron"
    \end{itemize}
\end{enumerate}

\begin{example}
    Alice notices that microscopes are not yet covered by any of the PMD ontologies. She opens a new issue titled \textit{"Microscopes are not covered yet"}. She also proposes a new definition \textit{"A microscope is a tool that makes things bigger"}. As she submits, the issue tracker assigns the issue number \textit{3141}.
\end{example}

\subsubsection{Working on open issues}

The ontology development takes place in {\github} issues. The main goal of these issues is to collect different opinions on the matter at hand and find a solution that reflects a consensus among all experts.

\begin{enumerate}
    \item \textbf{Check scope: } A lead developer evaluates whether the new concept fits into the scope definition of the ontology
    \item \textbf{Assign:} Assign a single ontology developer (further called \textit{assignee}) who is responsible for the implementation of this new concept.
    \item \textbf{Draft:} The assignee drafts a definition of the new concept and possible subsumption relations and axioms.
    \item \textbf{Discuss:} Other developers give feedback on these proposals and contribute to the discussion until a consensus is reached.
    \item \textbf{Agree or Talk:} If the discussion exceeds {\issuediscussionlimit} posts, it is moved to online meetings
\end{enumerate}

\todo{We need to set up a template repository}

\begin{example}
    Betty is one of the lead developers. They receive a notification of Alice's class request. They review the request and note that the concept is indeed missing from the current version of the ontology and within scope. They then decide that Claire will become the main assignee.

    Claire notes that the proposed definition is not precise enough, as it would also include other machines like tensile testing machines. She therefore proposes an alternative definition: \textit{"A microscope is a tool that magnifies objects."}

    Dillan, another ontology developer, points out that 'magnifying' is already in the ontology and that microscopes should be in relation to that class. 
    \[\mathrm{`microscope`} \sqsubseteq \exists \mathrm{`participates~in`}.\mathrm{`magnifying`}\]

    Claire points out that a microscope does not necessarily need to participate in a process in order to be a microscope and a discussion ensues. At this point, the discussion exceeds the threshold of {\issuediscussionlimit} posts and they decide to discuss this matter in the developer meeting and add issue number 1314 to the agenda.
\end{example}

\subsubsection{Working on open issues}

\begin{itemize}
    \item 
\end{itemize}

\begin{example}
In the next developer meeting, Betty, Claire and Dillan revisit issue 1314. During the discussion, they conclude that it is a function of microscopes to be used in magnifying processes. Therefore they decide to use the definition \emph{"A microscope is an object that can be used to magnify other objects"} and add the axiom
\emph{"`microscope` `has~function` \textbf{some} \`magnifying~function`"}.
The experts agree that they have reached a consensus and document their discussion in issue 1314. Dillan is tasked with the implementation of this issue.
\end{example}

\subsubsection{Implementing a change}
\begin{itemize}
    \item  Work should be done in a separate branch names "[IssueNumber]"
    \item add issue number to PR, use "closes \# issue number" to make {\github} close the issue once the PR is merged
    \item update changelog and/or term tracking
\end{itemize}

\begin{example}
Dillan has been tasked with implementing the consensus reached in the developer meeting. To do so, he pulls the most recent version from the development of the ontology and creates a new branch named \textit{1314\_add\_microscope}.

Then Dillan adds the necessary axioms to the ontology. He chooses to do so using {\protege}. He did already set up his working environment according Section~\ref{ssec:using-protege}. He opens the ontology in {\protege} and adds the missing class including its definition and the additional axiom.

Afterwards, he pushes his new branch to {\github} and opens a new Pull Request to merge his branch into the development branch.
\end{example}

\subsubsection{Reviewing a change}
Review the changes made on {\github} in the "files changed" section or commit-wise. Mark small change requests in the code directly or write a comment. Further, pull the proposed changes from the feature branch to protege and check there for inconsistencies. 

The following checklist can guide the review:
\begin{itemize}
    \item[$\Box$] Are classes and axioms placed correctly in the hierarchy?
    \item[$\Box$] Are spelling and grammar of definitions as well as axioms correct?
    \item[$\Box$] Are class definitions also reflected in the class axioms?
    \item[$\Box$] Are the classes classified correctly according to the BFO?
    \item[$\Box$] Are changes placed in the correct file, i.g. in case of a modular structure?
    \item[$\Box$] Are all steps of the requested changes done according to the issue and the expert decision?
    \item[$\Box$] Are there only changes that were previously agreed on in the discussion?
    \item[$\Box$] Are all changes made intentional? (e.g.Sometimes Protégé rearranges things which goes wrong in rare cases)
    \item[$\Box$] Are all relevant changes mentioned in the changelog / term trackers?
\end{itemize}
    
\begin{example}
Once Dillan is done with the implementation and pushed his proposed changes to {\github}, he marks the PR as "ready for review" and assigns Betty and Eric, a domain expert, as reviewers. 

Both start the review process by checking the questions of the reviewer's guide (list above). 
Betty has a change request, because Dillon missed to add "microscope" to the list of added terms in the changelog and marks this in the PR. Eric is content with the implementation and approves the PR, after Dillon added the note the the changelog.

After the approval and after the automated checks (CI) in {\github} were successful, Dillon merges the PR to the development branch. The term "microscope" has been added to the ontology.
\end{example}

\begin{description}
    \item[modules] Contains the actual ontology modules (T-box)
    \begin{itemize}
        \item Ontology files in turtle syntax
    \end{itemize}
    \item[patterns] Contains PMD Semantic Patterns that can be used to structure the ontology development and the shape definition
    \begin{itemize}
        \item LinkML
        \item SHACL
        \item OTTR
    \end{itemize}
    \item[shapes] General shapes that can be used to describe certain scenarios
    \begin{itemize}
        \item SHACL
    \end{itemize}
\end{description}


\section{Quality criteria}

\section{Tools for ontology development}

\begin{description}
    \item[Robot] Robot is a useful tool to process, translate ontologies, extract modules and information from and reason with ontologies. \url{http://robot.obolibrary.org/}
    \item[(py-)horned-owl] The rust library horned-owl and its python interface py-horned-owl are more modern interfaces to a general interface for ontology processing \url{https://github.com/phillord/horned-owl} and \url{https://github.com/jannahastings/py-horned-owl}
    \item[{\protege}] \url{https://protege.stanford.edu/}
\end{description}

\subsection{Using {\protege} for ontology development}
\label{ssec:using-protege}